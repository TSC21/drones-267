\section*{Motivation}

Currently, unmanned aerial vehicles (UAVs) are only partly autonomous. Existing
open-source, low-cost autopilots enable autonomous takeoff, flight, and
landing. However, the process of charging UAVs is still largely done by hand.
Because of the limited flight time of current UAVs, charging is a frequent
operation. Thus, the manual process of charging UAVs is a major barrier to
large-scale applications, such as farm monitoring.

One approach to automatic recharging of UAVs is to have them land on automated
charging stations. This requires that the UAV locate the charging station and
land accurately on top of it. Current methods for automated landing largely
rely on GPS localization, which does not provide sufficient accuracy. In the
best case, they get UAVs to land within a 3m by 3m box around the desired
landing location (3D Robotics, personal communication).

Our project aims to use on-board vision data to improve localization and
achieve more accurate landing. Several papers claim to solve this problem, but
there is little code available to the public that can be used with
off-the-shelf autopilots. Proprietary solutions do offer this functionality,
but they cost thousands of dollars and require deep technical expertise to
integrate.

Achieving autonomous landing will be a significant step towards building an
automated charging station. (The mechanical and electrical design of the
charging station are part of an ongoing MEng capstone project.) Our work will
be published under an open-source license.
